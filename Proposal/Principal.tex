%\documentclass[a4paper,12pt,oneside,openany]{article}
\documentclass[a4paper,12pt]{article}	
\input{FormatoPack}

\begin{document}

\begin{center}
\textbf{UNIVERSIDADE FEDERAL DO RIO DE JANEIRO}
\vspace{-0.2cm}

\textbf{ESCOLA POLITÉCNICA}
\vspace{-0.2cm}

\textbf{ENGENHARIA DE COMPUTAÇÃO E INFORMAÇÃO}
\vspace{0.8cm}

\underline{\textbf{PROPOSTA DE PROJETO DE GRADUAÇÃO}}

Aluno: Renan Fasolato Basilio
\vspace{-0.2cm}

renanbasilio@poli.ufrj.br

Orientador: Geraldo Zimbrão
\end{center}

\section{Título}

Streaming de Mídia Dinâmico-Adaptativo via WebSocket

\section{Ênfase}

Computação

\section{Tema}

O tema do trabalho é a implementação de um protocolo de streaming de mídia dinâmico-adaptativo sobre o protocolo WebSocket, e o estudo de suas vantagens e desvantagens em relação às tecnologias atualmente dominantes.

\section{Delimitação}

A realização do projeto estará restrita a mídias pré-fragmentadas e em container MPEG, estando fora do escopo do mesmo a análise e processamento de qualquer característica interna da mesma. Além disso, a maioria dos testes serão realizados em um ambiente simulado, podendo não refletir completamente todas as condições possíveis da rede real.

Por fim, a utilização do protocolo WebSocket para transferência dos dados implica na utilização 
do protocolo TCP, sobre o qual esse é implementado.

\section{Justificativa}

As tecnologias para streaming de mídia dominantes atualmente são baseadas no protocolo HTTP/S, sendo afetadas pelas desvantagens consolidadas do mesmo, dentre elas o consumo de banda com repetidos cabeçalhos do protocolo, a necessidade de reestabelecer conexão com o servidor (e, potencialmente, de refazer handshakes quando se trata de conexões seguras por SSL) para cada parte 
da mídia e a latência proveniente da necessidade de o cliente solicitar a próxima parte da mídia.

No entanto, com a recente estandardização do protocolo WebSocket, que permite o estabelecimento de uma conexão duplex reutilizável entre a página visualizada pelo cliente e o servidor remoto, e sua implementação na maioria dos browsers, torna-se viável uma nova solução para streaming de mídia baseada neste protocolo. A utilização deste protocolo permitirá cortar a maioria das latências provenientes do protocolo HTTP/S, e, por consequência, melhorando a capacidade de adaptação da mídia às variações na condição da rede.

Um trabalho semelhante existe em Wu et. al.\cite{Wu:2018}, porém, ainda que este utilize técnicas de Push para diminuir a latência, a responsabilidade da decisão sobre a necessidade de adaptação da stream permanece no cliente, o que implica em latência quando necessitar adaptação na mídia.


\section{Objetivo}

O objetivo deste trabalho é o desenvolvimento de um par cliente-servidor para streaming de mídia dinâmico-adaptativo através da internet utilizando o protocolo WebSocket. 

Essa implementação visa transferir a responsabilidade de decidir as características da mídia a ser enviada do cliente para o servidor, de forma que este possa transmitir um fluxo constante de dados sem qualquer latência entre dois fragmentos consecutivos.

\section{Metodologia}

O cliente terá a forma de uma biblioteca em JavaScript, utilizando a API Media Source Extensions para controlar um reprodutor de vídeo HTML5, e se comunicando com o servidor através de uma WebSocket. O cliente enviará mensagens de controle ao servidor para determinação de elementos como por exemplo a seleção da mídia a ser transmitida.

O servidor será desenvolvimento como um webapp em Java capaz de estabelecer conexão com um cliente, receber mensagens de controle deste, e, mediante análise heurística das condições da rede na comunicação com o mesmo, decidir as características do próximo fragmento da mídia a ser enviado ao cliente.

A mídia a ser transmitida será pré-processada a fim de fragmentá-la, tornando assim desnecessário qualquer processamento sobre a mesma por parte do servidor.

Para a realização dos testes será utilizada uma máquina virtual na qual a aplicação \textit{tc} será configurada para simular diversas condições de rede distintas. Esta escolha remete ao fato de que a utilização uma configuração simulada em cima de um sistema sem qualquer variação física resulta em dados altamente reprodutíveis, sem dependência das condições momentâneas da rede física.

\section{Materiais}

Para o desenvolvimento do sistema será utilizado software agnóstico ao sistema operacional, tornando o mesmo irrelevante a este projeto.As linguagens de programação utilizadas serão Java, JavaScript, HTML5 e CSS.

O ambiente de desenvolvimento será o Visual Studio Code, publicado sob a licença MIT Open Source, acompanhado do Apache Maven como toolchain de compilação, publicado sob a licença Apache 2.0 de software livre.

O servidor utilizado será o Apache Tomcat 9, também publicado sob a licença Apache de software livre.

Para configuração do ambiente de testes será utilizada a aplicação de virtualização Oracle VirtualBox, publicada sob a licença MIT Open Source. O sistema operacional tanto na máquina hospedeira quanto virtual será o Ubuntu Linux, publicado sob licença GPL share-alike, em suas versões Desktop (no hospedeiro) e Server (no servidor). Os testes serão realizados utilizando os browsers Mozilla Firefox e Google Chrome.

A mídia utilizada para os testes será o vídeo Big Buck Bunny, publicado pela Blender Foundation sob a licença GPL. A ferramenta FFmpeg, publicada sob a mesma licença, será utilizada para todo pré-processamento realizado sobre a mídia.

\section{Cronograma}

Fase 1: Desenvolvimento da Arquitetura do Cliente e Servidor

Fase 2: Desenvolvimento do Serviço de Streaming de Mídia

Fase 3: Desenvolvimento do Cliente para Extração de Dados

Fase 4: Configuração das Máquinas e Realização dos Testes de Desempenho

Fase 5: Produção de Relatório e Conclusões


\bibliographystyle{plain}
\bibliography{Bibliography}

\vspace{2cm}
\noindent
Rio de Janeiro, 27 de março de 2019

\vspace{1cm}
\begin{flushright}
      \parbox{10cm}{
            \hrulefill

            \vspace{-.375cm}
            \centering{Renan Fasolato Basilio - Aluno}

            \vspace{0.9cm}
            \hrulefill

            \vspace{-.375cm}
            \centering{Geraldo Zimbrão - Orientador}
 
            \vspace{0.9cm}
      }
\end{flushright}
\vfill
      
\end{document}
